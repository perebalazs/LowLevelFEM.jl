\documentclass[a4paper,11pt]{article}

\usepackage{amsmath,amssymb}
\usepackage{bm}
\usepackage{geometry}
\usepackage{hyperref}

\geometry{margin=2.5cm}

\title{Total Lagrange formuláció a LowLevelFEM-ben\\
	Gyenge alak és operátor-alapú implementáció}
\author{}
\date{}

\begin{document}
	\maketitle
	
	\section{Kiinduló probléma (Total Lagrange)}
	
	A referencia-konfigurációban ($\Omega_0$) felírt egyensúlyi egyenlet:
	\[
	\mathrm{Div}\,\bm P + \bm b_0 = \bm 0
	\quad \text{in } \Omega_0
	\]
	
	ahol
	\[
	\bm P = \bm F \bm S,
	\qquad
	\bm F = \bm I + \nabla_0 \bm u.
	\]
	
	Peremfeltételek:
	\[
	\bm u = \bar{\bm u} \quad \text{on } \Gamma_{u,0},
	\qquad
	\bm P \bm N = \bm t_0 \quad \text{on } \Gamma_{t,0}.
	\]
	
	---
	
	\section{Gyenge alak}
	
	A virtuális elmozdulás $\delta\bm u$ alkalmazásával:
	\[
	\int_{\Omega_0} \delta\bm F : \bm P \, \mathrm d\Omega_0
	=
	\int_{\Omega_0} \delta\bm u \cdot \bm b_0 \, \mathrm d\Omega_0
	+
	\int_{\Gamma_{t,0}} \delta\bm u \cdot \bm t_0 \, \mathrm d\Gamma_0,
	\]
	ahol
	\[
	\delta \bm F = \nabla_0 \delta \bm u.
	\]
	
	Ez az egyenlet a nemlineáris reziduál:
	\[
	\bm r(\bm u) = \bm f_{\mathrm{int}}(\bm u) - \bm f_{\mathrm{ext}}(\bm u) = \bm 0.
	\]
	
	---
	
	\section{Newton-iteráció}
	
	Newton-módszerrel:
	\[
	\bm K(\bm u^k)\,\Delta \bm u
	=
	\bm f_{\mathrm{ext}}(\bm u^k)
	-
	\bm f_{\mathrm{int}}(\bm u^k),
	\]
	
	ahol a teljes tangens:
	\[
	\bm K =
	\bm K_{\mathrm{mat}}
	+
	\bm K_{\mathrm{geo}}
	-
	\bm K_{\mathrm{ext}}.
	\]
	
	---
	
	\section{Operátor-alapú bontás a LowLevelFEM-ben}
	
	A LowLevelFEM-ben a gyenge alak térfogati és felületi tagjai
	különálló, "buta" operátorokként jelennek meg.
	A nemlinearitás kizárólag a mezőkön (pl.\ $\bm F$, $\bm S$, $\bm P$) keresztül lép be.
	
	\subsection{\texttt{poissonMatrixVector}}
	
	\paragraph{Gyenge alakbeli eredet}
	\[
	\int_{\Omega_0}
	\nabla_0 \delta \bm u : \bm S : \nabla_0 \bm u
	\,\mathrm d\Omega_0
	\]
	
	\paragraph{Szerep}
	\begin{itemize}
		\item geometriai merevség,
		\item II.~PK-feszültséggel súlyozott gradiens--gradiens tag,
		\item térfogati integrál.
	\end{itemize}
	
	\paragraph{Megjegyzés}
	A függvény teljes másodrendű tenzort (nem Voigt-formát) használ,
	így közvetlenül alkalmas nagy alakváltozásos feladatokra.
	
	---
	
	\subsection{\texttt{gradDivMatrixF}}
	
	\paragraph{Gyenge alakbeli eredet}
	\[
	\int_{\Omega_0}
	(\nabla_0 \cdot \delta \bm u)\,
	\lambda\,
	(\nabla_0 \cdot \bm u)
	\,\mathrm d\Omega_0,
	\quad
	\text{deformált konfigurációhoz húzva } \bm F\text{-fel}.
	\]
	
	\paragraph{Szerep}
	\begin{itemize}
		\item anyagi tangens volumetrikus része,
		\item izotrop anyag $\lambda$ paramétere,
		\item $\bm F$-függő (nagy alakváltozás).
	\end{itemize}
	
	---
	
	\subsection{\texttt{poissonMatrixSymGradF}}
	
	\paragraph{Gyenge alakbeli eredet}
	\[
	\int_{\Omega_0}
	2\mu\,
	\delta \bm E : \bm E
	\,\mathrm d\Omega_0,
	\qquad
	\bm E = \tfrac12(\bm F^T \nabla_0 \bm u + \nabla_0 \bm u^T \bm F).
	\]
	
	\paragraph{Szerep}
	\begin{itemize}
		\item anyagi tangens deviátoros része,
		\item izotrop anyag $\mu$ paramétere,
		\item szimmetrikus Green--Lagrange-alapú forma.
	\end{itemize}
	
	---
	
	\subsection{\texttt{internalForceTL}}
	
	\paragraph{Gyenge alakbeli eredet}
	\[
	\bm f_{\mathrm{int}}
	=
	\int_{\Omega_0}
	\bm B^T \bm P
	\,\mathrm d\Omega_0
	=
	\int_{\Omega_0}
	\nabla_0 \bm N : \bm P
	\,\mathrm d\Omega_0.
	\]
	
	\paragraph{Szerep}
	\begin{itemize}
		\item nemlineáris belső erővektor,
		\item minden Newton-iterációban újraszámolandó,
		\item térfogati integrál.
	\end{itemize}
	
	\paragraph{Megjegyzés}
	A First Piola--Kirchhoff feszültség mezőként kerül be,
	így az anyagmodell teljesen kívül marad.
	
	---
	
	\subsection{\texttt{externalTangentFollowerTL}}
	
	\paragraph{Gyenge alakbeli eredet}
	Follower terhelés esetén:
	\[
	\bm t_0 = J \bm F^{-T} \bm t,
	\]
	
	\[
	\bm K_{\mathrm{ext}}
	=
	\int_{\Gamma_{t,0}}
	\bm N^T
	\frac{\partial (J\bm F^{-T})}{\partial \bm u}
	\bm t
	\,\mathrm d\Gamma_0.
	\]
	
	\paragraph{Szerep}
	\begin{itemize}
		\item külső erők konzisztens Newton-tangense,
		\item kizárólag felületi integrál,
		\item csak follower load esetén nem nulla.
	\end{itemize}
	
	\paragraph{Megjegyzés}
	Dead load esetén ez a mátrix zérus,
	és a meglévő \texttt{loadVector} elegendő.
	
	---
	
	\section{Összefoglaló táblázat}
	
	\[
	\begin{array}{l|c|c|c}
		\text{LowLevelFEM függvény} &
		\text{Gyenge alak} &
		\text{Integrál} &
		\text{Newton-tag} \\
		\hline
		\texttt{poissonMatrixVector} & \bm K_{\mathrm{geo}} & \Omega_0 & igen \\
		\texttt{gradDivMatrixF} & \bm K_{\mathrm{mat}} & \Omega_0 & igen \\
		\texttt{poissonMatrixSymGradF} & \bm K_{\mathrm{mat}} & \Omega_0 & igen \\
		\texttt{internalForceTL} & \bm f_{\mathrm{int}} & \Omega_0 & igen \\
		\texttt{externalTangentFollowerTL} & \bm K_{\mathrm{ext}} & \Gamma_{t,0} & igen \\
	\end{array}
	\]
	
	---
	
	\section{Záró megjegyzés}
	
	A bemutatott struktúra lehetővé teszi, hogy:
	\begin{itemize}
		\item kis és nagy alakváltozás azonos operátorokra épüljön,
		\item az anyagmodell és a geometria szétváljon,
		\item a Newton-iteráció konzisztens maradjon.
	\end{itemize}
	
	A LowLevelFEM filozófiája szerint az operátorok
	\emph{minimális tudásúak},
	míg a fizikai nemlinearitás a mezőkön keresztül jelenik meg.
	
\end{document}
