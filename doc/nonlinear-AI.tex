% !TEX program = pdflatex
\documentclass[11pt,a4paper]{article}
\usepackage{amsmath,amssymb,bm}
\usepackage{geometry}
\geometry{margin=2.5cm}

% --- Notation helpers ---
\newcommand{\R}{\mathbb{R}}
\newcommand{\Grad}{\nabla_0}   % reference gradient
\newcommand{\Div}{\nabla_0\cdot} % reference divergence
\newcommand{\F}{\bm F}
\newcommand{\C}{\bm C}
\newcommand{\E}{\bm E}
\newcommand{\Ssig}{\bm S}
\newcommand{\Pii}{\bm P}
\newcommand{\I}{\bm I}
\newcommand{\tr}{\mathrm{tr}}
\newcommand{\dd}{\,\mathrm d}
\renewcommand{\v}{\bm v}
\newcommand{\uvec}{\bm u}
\newcommand{\deltau}{\Delta\bm u}
\newcommand{\etaV}{\delta\bm v} % virtual displacement (test function)

\title{Nagy alakváltozású szilárdságtan: gyenge alak és Newton-iteráció (Total Lagrange)}
\author{}
\date{}

\begin{document}
	\maketitle
	
	\section{Kinematika és feszültségmértékek}
	
	Legyen a referencia konfiguráció $\Omega_0\subset\R^3$ határával $\partial\Omega_0=\Gamma_{0u}\cup\Gamma_{0t}$.
	A leképezés:
	\[
	\bm x = \bm \chi(\bm X) = \bm X + \uvec(\bm X),
	\qquad \bm X\in\Omega_0.
	\]
	Deformációgradiens:
	\[
	\F(\uvec) := \frac{\partial \bm x}{\partial \bm X} = \I + \Grad \uvec.
	\]
	Jobb Cauchy--Green-tenzor és Green--Lagrange alakváltozási tenzor:
	\[
	\C = \F^{T}\F,
	\qquad
	\E = \frac{1}{2}(\C-\I).
	\]
	
	Total Lagrange leírásban tipikusan a II. Piola--Kirchhoff feszültséget használjuk:
	\[
	\Ssig = \Ssig(\E).
	\]
	(Hiperelaszticitás esetén $\Ssig = 2\,\partial\psi/\partial\C$, és a konzisztens anyagi tangens $\mathbb C := \partial\Ssig/\partial\E$.)
	
	\section{Erőegyensúly és peremfeltételek (referencia konfigurációban)}
	
	A referencia konfigurációban a test-erő sűrűség legyen $\bm b_0$ (N/m$^3$),
	a kijelölt traction peremen a névleges felületi terhelés legyen $\bm t_0$ (N/m$^2$).
	A peremfeltételek:
	\[
	\uvec=\bar{\uvec}\ \text{a } \Gamma_{0u}\text{-n}, 
	\qquad
	\Pii\,\bm N = \bm t_0\ \text{a } \Gamma_{0t}\text{-n},
	\]
	ahol $\bm N$ a referencia konfiguráció külső egységnormálisa, és az I. Piola--Kirchhoff feszültség:
	\[
	\Pii = \F\,\Ssig.
	\]
	
	\section{Gyenge alak (virtuális munka elve)}
	
	Válasszunk tesztfüggvényt $\etaV\in \mathcal V_0$, ahol
	\[
	\mathcal V_0 := \{\etaV:\Omega_0\to\R^3\mid \etaV=\bm 0\ \text{a } \Gamma_{0u}\text{-n}\}.
	\]
	A belső és külső virtuális munka:
	\[
	\delta W_{\mathrm{int}}(\uvec;\etaV)
	:= \int_{\Omega_0} \Pii(\uvec) : \Grad \etaV \dd V,
	\]
	\[
	\delta W_{\mathrm{ext}}(\etaV)
	:= \int_{\Omega_0} \etaV\cdot \bm b_0 \dd V
	+ \int_{\Gamma_{0t}} \etaV\cdot \bm t_0 \dd A.
	\]
	A gyenge alak:
	\begin{equation}
		\boxed{
			\delta W(\uvec;\etaV)
			:= \delta W_{\mathrm{int}}(\uvec;\etaV) - \delta W_{\mathrm{ext}}(\etaV) = 0
			\quad \forall\,\etaV\in\mathcal V_0.
		}
		\label{eq:weak}
	\end{equation}
	
	\paragraph{Megjegyzés (terheléstípus).}
	A fenti $\bm t_0$ és $\bm b_0$ \emph{referencia konfigurációhoz kötött} (``dead load'').
	Ha a terhelés a jelenlegi konfigurációt követi (``follower load''), akkor $\delta W_{\mathrm{ext}}$ linearizációjában további tagok jelennek meg.
	
	\section{Newton--Raphson iteráció: reziduum és konzisztens linearizáció}
	
	A \eqref{eq:weak} egyenletből definiáljuk a nemlineáris operátort (reziduumot):
	\[
	R(\uvec)[\etaV] := \delta W(\uvec;\etaV).
	\]
	A feladat: találni $\uvec\in\mathcal U_{\bar u}$-t úgy, hogy
	\[
	R(\uvec)[\etaV]=0 \quad \forall\,\etaV\in\mathcal V_0,
	\]
	ahol $\mathcal U_{\bar u}$ a Dirichlet-peremfeltételt kielégítő függvénytér.
	
	\subsection{Newton-lépés}
	
	Adott $k$ iterációban $\uvec^k$ mellett keressük az inkrementumot $\deltau$, hogy
	\begin{equation}
		\boxed{
			DR(\uvec^k)[\deltau,\etaV] = -R(\uvec^k)[\etaV]
			\quad \forall\,\etaV\in\mathcal V_0,
		}
		\label{eq:newton}
	\end{equation}
	majd frissítünk:
	\[
	\uvec^{k+1} = \uvec^k + \deltau.
	\]
	
	\subsection{A belső virtuális munka linearizációja (anyag + geometria)}
	
	Mivel a külső munka (dead load esetén) nem függ $\uvec$-től, ezért
	\[
	DR(\uvec)[\deltau,\etaV] = D\delta W_{\mathrm{int}}(\uvec;\etaV)[\deltau].
	\]
	A belső tag:
	\[
	\delta W_{\mathrm{int}}(\uvec;\etaV)=\int_{\Omega_0} \Pii(\uvec):\Grad\etaV\dd V
	=\int_{\Omega_0} (\F(\uvec)\Ssig(\uvec)):\Grad\etaV\dd V.
	\]
	Szorzatszabállyal:
	\[
	D\Pii[\deltau] = D(\F\Ssig)[\deltau]
	= (D\F[\deltau])\,\Ssig + \F\, (D\Ssig[\deltau]).
	\]
	Továbbá
	\[
	D\F[\deltau]=\Grad\deltau.
	\]
	
	\paragraph{Anyagi rész.}
	A $D\Ssig$ kinematikán keresztül:
	\[
	D\Ssig[\deltau] = \mathbb C : D\E[\deltau],
	\qquad
	\mathbb C := \frac{\partial \Ssig}{\partial \E}.
	\]
	A Green--Lagrange alakváltozás variációja:
	\[
	D\E[\deltau]
	= \frac{1}{2}\left(D\F^{T}[\deltau]\F + \F^{T}D\F[\deltau]\right)
	= \frac{1}{2}\left((\Grad\deltau)^{T}\F + \F^{T}\Grad\deltau\right).
	\]
	
	\paragraph{Geometriai rész.}
	A geometriai rész a $(D\F)\Ssig$ tagból származik.
	
	\subsection{Konzisztens bilineáris forma (tangens)}
	
	Ezekkel a Newton-baloldal (tangens) bilineáris formában:
	\begin{equation}
		\boxed{
			DR(\uvec)[\deltau,\etaV]
			=
			\underbrace{\int_{\Omega_0} \left(\Grad\deltau\,\Ssig\right):\Grad\etaV\dd V}_{\text{geometriai (initial stress) rész}}
			+
			\underbrace{\int_{\Omega_0} \left(\F\,(\mathbb C : D\E[\deltau])\right):\Grad\etaV\dd V}_{\text{anyagi (material) rész}}.
		}
		\label{eq:tangent}
	\end{equation}
	
	\paragraph{Ekvivalens alak (gyakran használt).}
	A második tagban be lehet vezetni a \emph{virtuális} és \emph{inkrementális} Green--Lagrange alakváltozásokat:
	\[
	\delta\E(\etaV) = \frac{1}{2}\left((\Grad\etaV)^{T}\F + \F^{T}\Grad\etaV\right),
	\qquad
	\Delta\E(\deltau) = \frac{1}{2}\left((\Grad\deltau)^{T}\F + \F^{T}\Grad\deltau\right),
	\]
	amivel az anyagi rész sok anyagtörvény mellett írható:
	\[
	\int_{\Omega_0} \delta\E(\etaV) : \mathbb C : \Delta\E(\deltau)\dd V
	\]
	(ekkor figyelni kell az index-konvenciókra és arra, hogy $\mathbb C$ negyedrendű tenzor).
	
	\subsection{Newton-algoritmus (pszeudokód)}
	
	\begin{enumerate}
		\item Válassz kezdeti tippet: $\uvec^0$.
		\item $k=0,1,2,\dots$
		\begin{enumerate}
			\item Számítsd ki a reziduumot: $R(\uvec^k)[\etaV]$.
			\item Állítsd elő a tangens bilineáris formát: $DR(\uvec^k)[\deltau,\etaV]$ a \eqref{eq:tangent} szerint.
			\item Oldd meg: $DR(\uvec^k)[\deltau,\etaV] = -R(\uvec^k)[\etaV]\ \ \forall\etaV$.
			\item Frissítés: $\uvec^{k+1}=\uvec^k+\deltau$.
			\item Konvergencia-ellenőrzés (pl. $\|\deltau\|$ vagy $\|R\|$ alapján).
		\end{enumerate}
	\end{enumerate}
	
\end{document}
